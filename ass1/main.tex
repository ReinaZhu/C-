%%%%%%%%%%%%%%%%%%%%%%%%%%%%%%%%%%%%%%%%%
% Wenneker Assignment
% LaTeX Template
% Version 2.0 (12/1/2019)
%
% This template originates from:
% http://www.LaTeXTemplates.com
%
% Authors:
% Vel (vel@LaTeXTemplates.com)
% Frits Wenneker
%
% License:
% CC BY-NC-SA 3.0 (http://creativecommons.org/licenses/by-nc-sa/3.0/)
%
%%%%%%%%%%%%%%%%%%%%%%%%%%%%%%%%%%%%%%%%%

%----------------------------------------------------------------------------------------
%	PACKAGES AND OTHER DOCUMENT CONFIGURATIONS
%----------------------------------------------------------------------------------------

\documentclass[11pt]{scrartcl} % Font size

\input{structure.tex} % Include the file specifying the document structure and custom commands

%----------------------------------------------------------------------------------------
%	TITLE SECTION
%----------------------------------------------------------------------------------------

\title{	
	\normalfont\normalsize
	\textsc{Southern University of Science and Technology}\\ % Your university, school and/or department name(s)
	\vspace{25pt} % Whitespace
	\rule{\linewidth}{0.5pt}\\ % Thin top horizontal rule
	\vspace{20pt} % Whitespace
	{\huge C/C++ Programming Language assignment 1}\\ % The assignment title
	\vspace{12pt} % Whitespace
	\rule{\linewidth}{2pt}\\ % Thick bottom horizontal rule
	\vspace{12pt} % Whitespace
}

\author{\LARGE Zhu Yujie 11712940} % Your name

\date{\normalsize\today} % Today's date (\today) or a custom date

\begin{document}
\noindent

\maketitle % Print the title

%----------------------------------------------------------------------------------------
%	FIGURE EXAMPLE
%----------------------------------------------------------------------------------------

\section{Analysis}

    The problem is to calculate the distance between two places by their latitude and longitude. As we all know that the radius of earth is 6371 km, assume earth is a perfect sphere, we could use latitude and longitude to calculate the angle. Firstly, compute $\phi$:
    \par
    \centerline{\large $\phi$=(90-latitude)$\times$ $\pi$ $\div$180}
    \par Where latitude is in angle.
    \par And let longitude be in radian too.
    \par
    \centerline{\large $\theta$=longitude$\times$ $\pi$ $\div$180}
    \par Where longitude is also in angle.
    \par
    Using the following formula could compute the shortest distance between the two places.
    \par
    \centerline{\large $c$=$\sin$($\phi$ $_1$)$\times$ $\sin$($\phi$ $_2$)$\times$ $\cos$($\theta$ $_1$ - $\theta$ $_2$)+$\cos$ ($\phi$ $_1$)$\times$ $\cos$ ($\phi$ $_2$)  }
    \centerline{\large $d$=R$\times$ $\arccos$($c$)}
    \par Where R is the radius of earth that is R = 6371 km.



%------------------------------------------------


%----------------------------------------------------------------------------------------
%	TEXT EXAMPLE
%----------------------------------------------------------------------------------------

\section{Code}
\lstinputlisting[
	caption=Assignment 1., % Caption above the listing
	frame=single, % Frame around the code listing
	showstringspaces=false, % Don't put marks in string spaces
	numbers=left, % Line numbers on left
	numberstyle=\tiny, % Line numbers styling
	]{assignment1.cpp}
\vspace{20pt}

%\begin{figure}[htb] % [h] forces the figure to be output where it is defined in the code (it suppresses floating)
%	\centering
%	\includegraphics[width=150mm]{code.jpg} % Example image
%	\caption{code.}
%\end{figure}

%------------------------------------------------


%----------------------------------------------------------------------------------------
%	EQUATION EXAMPLES
%----------------------------------------------------------------------------------------
\newpage
\section{ Result $\&$ Verification}

\vspace{20pt}

{Test case $\#$1:}
\lstinputlisting[
	caption=Test1., % Caption above the listing
	frame=single, % Frame around the code listing
	showstringspaces=false, % Don't put marks in string spaces
	numbers=left, % Line numbers on left
	numberstyle=\tiny, % Line numbers styling
	]{case1.txt}
\vspace{20pt}
\begin{figure}[htb] % [h] forces the figure to be output where it is defined in the code (it suppresses floating)
	\centering
	\includegraphics[width=100mm]{case11.png} % Example image
	\caption{test1.}
\end{figure}
\vspace{20pt}
{Test case $\#$2:}
\lstinputlisting[
	caption=Test2., % Caption above the listing
	frame=single, % Frame around the code listing
	showstringspaces=false, % Don't put marks in string spaces
	numbers=left, % Line numbers on left
	numberstyle=\tiny, % Line numbers styling
	]{case2.txt}
\vspace{20pt}
\begin{figure}[htb] % [h] forces the figure to be output where it is defined in the code (it suppresses floating)
	\centering
	\includegraphics[width=100mm]{case22.png} % Example image
	\caption{test2.}
\end{figure}






\section{Difficulties $\&$ Solutions}
\begin{enumerate}
  \item To understand what the formula and model means.
  \item To handle unexpected input, without reporting error and ask user to input again.
  \item Some cities’ names contain two or more words but store in one variable.
\end{enumerate}

%----------------------------------------------------------------------------------------

\end{document}
